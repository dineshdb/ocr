\chapter{Conclusion}

In this project, an OCR system has been developed which employs Maximally Stable Extremal Regions as basic letter candidates. To overcome the sensitivity of MSER with respect to image blur and to detect even very small letters, a modified version of MSER complimented with the Canny edges is used. The detected text are binarized letter patches, which are directly used for text recognition purposes. CNN does not require specific feature to be added as input. It works on the raw pixel of image and extracts features from there based on the number of layers and the number of neurons in each layers. The system worked with an accuracy of around 83\% which makes us confident for using it in recognizing handwritten as well as printed texts.

\section{Limitations}
\begin{enumerate}
\item Accuracy of the system is not satisfactory.
\end{enumerate}

\section{Future Enhancements}
\begin{enumerate}
\item Accuracy of the system can be improved by using a much larger dataset.
\item Second, the CNN architecture can be improved further to include more features.
\item Pre-processing of image can be improved using techniques like stroke width transforms which provides a much greater confidence in identifying text regions in the image.
\item Extension of the system to identify whole words at a time instead of characters
\item Can be extended to other languages, logos, symbols, and even used in vehicle plate recognition.
\end{enumerate}
